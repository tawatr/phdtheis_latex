
% !TEX root = ../thesis_phd.tex 

\chapter{Unused topics and materials}
	\red{Check whether the definition of static shift is given}.
	
	\section{Intro}
\begin{itemize}
	\item \redb{(The theoretical model of regional mean 1-D profile was examined in Section XXX) [Have no idea where to place it]}
	\item The choices of choosing the initial and also \emph{a priori} models yet remain controversial. 
	\item \red{Writing the conductivity distribution in this way, the anomaly is represented by the conductivity contrast.}
	\item The regional structure is changed with changing frequency range.
	\item However, the definition of the regional conductivity profile, $\sigma_\Regional$, is never explicitly given. In this work we present the definition to calculate the theoretical model of regional mean 1-D conductivity profile.
		The small-scale structure confined in near-surface thin, thinner than the inductive scale length, layers is considered as distorter \citep{utada2000a}.
	\item In this work, the regional scale means the structure which its horizontal dimension is comparable to or larger than the inductive scale length of present interest \citep[also see][]{bahr1988a}. 
	\item \citet{tournerie2002a} also use average impedance to estimate the good starting model for 2D inversions	
\end{itemize}
	\section{Galvanic distortion}
	\begin{itemize}
		\item In this context, Small and spatial dependent  $\rightarrow$ assumed random 
		\item Although, the formulation of Groom-Bailey's model was criticized controversial and not intuitively obvious \citep{bibby2005a}, it is \emph{mathematically} possible.
	\end{itemize}	
	\section{Synthetic tests}
	%% ==== Table: Layered-earth model
\begin{table}
	\centering
	\begin{tabular}{ccc}
	\hline 
	Layer & Depth (km) & Resistivity (\Ohmm) \\ \hline
	Surface & 0.0 -- 3.5 & 100 \\
	Upper crust & 3.5 -- 14.8 & 1000 \\
	Lower crust & 14.8 -- 33.3 & 30 \\
	Upper mantle & 33.3 -- 136.0 & 100 \\
	Asthenosphere & 136.0 -- 203.0 & 10 \\
	Upper mantle & 203.0 -- 451.0 & 30 \\
	Mantle transition zone & 451.0 -- 673.0 & 3 \\
	Lower mantle & 673.0 -- $\infty$ & 1 \\ \hline
	\end{tabular}
	\caption{The detail of the layered earth in Figure \ref{fig:example1d_model}.}
	\label{tab:example1d_model}
\end{table}
\subsection{Synthetic 1D}
%% ====
 \red{-- PENDING --}
\begin{itemize}
	\item \red{Add the results showing that if we use the individual (distorted) observation to represent the regional structure, the conductivity contrast may not be minimized.}
\end{itemize}
\subsection{Synthetic 3D}
\begin{itemize}
	\item Add the results showing that if we use the response from different underlying structure may have different
	\item The distorted det impedances in the synthetic tests (Figures \ref{fig:resp1d_individual_all_distorted_sd3a_det} and \ref{fig:resp3d_individual_all_distorted_sd3a_det}) also resemble the results in  \citet{berdichevsky1980a}.	
	
\end{itemize}

\begin{itemize}
	\item The average approach helps flatten the contribution from underlying 3-D structure
\end{itemize}
